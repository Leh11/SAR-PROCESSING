
\documentclass{article}
\usepackage[document]{ragged2e}
\usepackage[portuguese]{babel}
\usepackage{booktabs}
\usepackage[a4paper, left=3cm, right=2cm, top=3cm, bottom=2cm]{geometry}
\usepackage{amssymb,amsmath}

\date{}
\title{RESUMO}
\begin{document}
\maketitle

\section{Decomposição de Barnes-Holme}

Partindo do pre-suposto de que a decomposição de \textbf{Hyunen} fatoriza a resposta polarimétrica \textit{Matriz de Coerência} em \textit{$T_{0}$ - alvos puros} contendo cinco parâmetros \textit{$rank - 1$} e, \textit{$T_{N}$ - em alvos distribuídos} com resto dos parâmetros \textit{$rank > 1$} que é dado como \textit{roll invariant} ao ângulo de aquisição. Assim sendo é plausível a interpretação de que o espaço vetorial gerado pela \textit{$T_{0}$} deve ser ortogonal ao espaço gerado pela \textit{$T_{0}$}. Para satisfazer estas condições, introduziu-se pelo \textbf{Barnes-Holm} um vetor $q$ que ao multiplica-lo  pela $T_{N}$ gera um vetor nulo; ou seja $T_{0}q = 0$. Com isso, permanece a ideia da ortogonalidade e a invariabilidade em relação ao ângulo, independentemente da rotação da polarização. Segue a normalização e a composição da equações:

Dada a matriz $U$, definida pelo

    \begin{equation}
        U = \begin{bmatrix}
            1 & 0 & 0 \\
            0 & \cos{2\theta} & \sin{2\theta} \\
            0 & -\sin{2\theta} & \cos{2\theta}
        \end{bmatrix}
    \end{equation} 
Obtém-se o vetor $q$ como sendo o autovetor de $U$, decomposto em que $q_{1}, q_{2} \ e \ q_{3}$ em que

    \begin{equation}
        q_{1} =\begin{bmatrix}
            1 \\
            0 \\
            0
        \end{bmatrix} \ 
        q_{2} = \frac{1}{\sqrt{2}}\begin{bmatrix}
            0 \\
            1 \\
            j
        \end{bmatrix} \ 
        q_{3} = \frac{1}{\sqrt{2}}\begin{bmatrix}
            0 \\
            j \\
            1
        \end{bmatrix}
    \end{equation}
    
Como proposto pelo \textbf{Hyunen}, a matriz de coerência $T_{3}$ pode ser decomposto em \textit{alvos puros} e \textit{alvos distribuidos} de três formas diferentes, usando os três autovetores acima citadas, propostos pelo \textbf{Barnes-Holm}. 
Tomando cada autovetor, pode-se normalizar um vetores como segue:

\begin{equation}
    k_{01} = \frac{T_{3}q_{1}}{\sqrt{q_{1}^{T*}T_{3}q_{1}}} \    
    k_{02} = \frac{T_{3}q_{2}}{\sqrt{q_{2}^{T*}T_{3}q_{2}}} \    
    k_{03} = \frac{T_{3}q_{3}}{\sqrt{q_{3}^{T*}T_{3}q_{3}}} \ 
\end{equation}

O primeiro vetor corresponde exatamente a formulação feita pelo \textbf{Hyunen}, o que permite considerar somente os dois últimos neste teorema \cite{jong:2009}.

%\section{Decomposição}
\bibliographystyle{unsrt}
\bibliography{ref}
\end{document}
