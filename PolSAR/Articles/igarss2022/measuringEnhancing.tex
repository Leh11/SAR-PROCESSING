    % Template for IGARSS-2020 paper; to be used with:
    %          spconf.sty  - LaTeX style file, and
    %          IEEEbib.bst - IEEE bibliography style file.
    % --------------------------------------------------------------------------
    \documentclass{article}
    \usepackage{spconf,amsmath,epsfig}
    \usepackage{bbm,bm}
    \usepackage{natbib}
    
    % Example definitions.
    % --------------------
    \def\x{{\mathbf x}}
    \def\L{{\cal L}}
    
    % Title.
    % ------
    \title{Measuring and Enhancing the Visual Content of Polarimetric Synthetic Aperture Radar Decompositions}
    %
    % Single address.
    % ---------------
    %\name{Author(s) Name(s)\thanks{Thanks to XYZ agency for funding.}}
    %\address{Author Affiliation(s)}
    %
    % For example:
    % ------------
    %\address{School\\
    %	Department\\
    %	Address}
    %
    % Two addresses (uncomment and modify for two-address case).
    % ----------------------------------------------------------
    \twoauthors
      {Alejandro C.\ Frery\sthanks{CONTACT A.\ C.\ Frery Email: alejandro.frery@vuw.ac.nz}}
    	{Victoria University of Wellington, New Zealand\\
    	School of Mathematics and Statistics}
      {Ulil\'e Indeque}
    	{Universidade Federal de Alagoas, Brazil\\
    	Laborat\'orio de Computa\c c\~ao Cient\'ifica \\e An\'alise Num\'erica}
    	
    \begin{document}
    %\ninept
    %
    \maketitle
    %
    \begin{abstract}
    Polarimetric Synthetic Aperture Radar images disposes an amount of important information about the iluminated scene. These images contains many properties that can explored. The CIELab is an color space which presents an approximately uniform distribution and closer to human perception. To take advantage of it, image manipulation are performed with images encoded in CIELab color space. This paper aims to presents the technique to enhacing the image quality using matching histogram, manipulating the L component. The ideal image used as reference was generated randomly and has a uniform distribution. The observed images were generated by applying the Barnes-Holmes, Pauli, Krogager, and Hyunen decomposition algorithms.
    By the way of the information content comparison, the Hellinger Distance which is an important metric in the fields of information theory is used to measure the distance between two probability distributions, which are: the L component histogram of the ideal and original image. 
    
    \end{abstract}
    %
    \begin{keywords}
    Image enhancement;
    Information content;
    Polarimetric decompositions;
    SAR Polarimetry;
    Stochastic distances.
    \end{keywords}
    %
    \section{Introduction}
    \label{sec:intro}
    
    Polarimetric Synthetic Aperture Radar (PolSAR) images have a prominent role in remote sensing \citep{HolarisationApplicationsRemoteSensing,LeePottier2009PolarimetricRadarImaging}.
    Such images are formed by the return from the scene in several combinations of transmitting and receiving polarization of the electromagnetic waves.
    
    The observation in each pixel of a fully polarimetric image is a $2\times2$ complex-valued matrix:
    \begin{equation}
    \bm S' = \begin{pmatrix}
    	S_{\text{HH}} & S_{\text{HV}}\\
    	S_{\text{VH}} & S_{\text{VV}}
    \end{pmatrix}, \label{Eq:ScatteringMatrix}
    \end{equation}
    in which $S_{ij}$ is a complex value, and $i$ and $j$ are any of the two possible orientations: $\text{H}$ (horizontal) or $\text{V}$ (vertical).
    Under the reciprocity principle, $S_{\text{HV}}=S_{\text{VH}}$, so the information in the scattering matrix~\eqref{Eq:ScatteringMatrix} can be encoded in the scattering vector
    \begin{equation}
    \bm S = \begin{pmatrix}
    	S_{\text{HH}}\\
    	S_{\text{HV}}\\
    	S_{\text{VV}}
    \end{pmatrix}
    \end{equation}
    without loss of information.
    
    More often than not, users employ multilooked images, in which $L$ ideally independent observations are averaged:
    \begin{equation}
    \bm Z = \frac{1}{L} \sum_{\ell=1}^L \bm S(\ell) \bm S^\dag(\ell),
    \end{equation}
    where ``$\dag$'' denotes the transpose of the complex conjugate.
    With this, the multilook observation in each pixel has the form
    \begin{equation}
    \bm Z = \begin{pmatrix}
    I_{\text{HH}} & \text{Cov}(S_{\text{HH}},S_\text{HV}) & \text{Cov}(S_{\text{HH}},S_\text{VV}) \\
    \text{Cov}(S_{\text{HH}},S_\text{HV})^* & I_{\text{HV}} & \text{Cov}(S_{\text{HV}},S_\text{VV})\\
    \text{Cov}(S_{\text{HH}},S_\text{VV})^* & \text{Cov}(S_{\text{HV}},S_\text{VV})^* & I_{\text{VV}}
    \end{pmatrix},
    \label{Eq:Multilook}
    \end{equation}
    in which ``$*$'' denotes the complex conjugate, the diagonal elements are intensities, and the off-diagonal entries are covariances.
    
    It is not possible to visualize directly observations of the form~\eqref{Eq:Multilook}, as they belong to $\mathbbm R_+^3\times \mathbbm C^3$, i.e., they are nine-dimensional objects.
    Such observations encode valuable information from the target, and it is desirable to extract and visualize it.
    Such extraction can be accomplished using Theorem Decompositions.
    
    \subsection{Decompositions Theorems}
    
    PolSAR decomposition consists in retrieving the components which gave rise to each observation.
    The purpose of such operation is to identify the average scattering mechanism within each pixel.
    In the following, we will summarize the main theorems for incoherent observations.
    
    \citet{ModelingandInterpretationofScatteringMechanismsinPolarimetricSyntheticApertureRadarAdvancesandPerspectives2014} discuss the general additive decomposition framework
    \begin{equation}
    \bm Z = \sum_i^N \pi_i \bm Z^{(i)},
    \end{equation}
    in which $\bm Z^{(i)}$ is a prototypical scattering, 
    $\pi_i$ the relative amount with which it contributes to the observation $\bm Z$, and $N$ is the number of components in the resolution cell.
    Alternatively, \citet{APolSARScatteringPowerFactorizationFrameworkandNovelRollInvariantParametersBasedUnsupervisedClassificationSchemeUsingaGeodesicDistanceinpress} discuss a multiplicative rather than additive approach to this problem.
    
    There are several well-known prototypical scatterers, whose form depends on the target.
    Among them, we mention the volume, double bounce, odd bounce, and helix.
    Their coefficients in an observation depends both on the target and on the sensing system conditions.
    
    We are interested in decomposition theorems that naturally lead to visualization, i.e., that produce three components that can be mapped onto the red, green, and blue channels: Barnes-Holmes, Pauli, Krogager, and Hyunen.
    
    \subsection{Information Content}
    
    \citet{AssessingInformationContentinColorImages} proposed a technique for measuring the information content of color images with the Kullback-Leibler divergence between the observed and ideal histograms of the pixel values in the CIELAB color space.
    This approach also allows improving an image's contrast by maximizing its realizable color information while retaining the chroma and hue.
    
    
    \section{METHODOLOGY}
    \label{sec:meth}
    In this section, we will talk about the proposed technique to enhacing the images and the metric to assess the image quality.
    
    The ideal image is an image geneted randonly that contain an uniform distribuition and mapped in CIELab space, denoted by $I=(L_I,a_I,b_I)$, where \textbf{L} represent the luminance component with pixel distribuited from 0 - more dark to 100 - more clear; \textbf{a} component represent green color in the negative axis and red in the positive and, \textbf{b} component represent blue color in the negative axis, whereas yellow in the positive. The a and b components vary from -200 to 200.
    
    The originals image are denoted by $O=(r_O, g_O, b_O)$ and encoded in the RGB color space. These images are converted to CIELab color space having the components $O=(L_O, a_O, b_O)$ and, in th final of operation, they are inversely converted to RGB color space, since they are inversible. The mapping from RGB to CIELab can be found in \citep{AssessingInformationContentinColorImages}. 
    
    \subsection{Proposed approach}
    The matching histogram technique consists in adjust the histogram of an image to be similar to another referenced image.
    
    Consider the two above mentioned object $I$ and $O$, respectively the ideal and original image. The matching histogram function receive two object: the $H_{L_I}$ component as reference and the $H_{L_O}$ interest's object. 

    The first step is equalization of $H_{L_O}$. This is done by \textbf{\textit{"cumulative density function - CDF"}}, as follow
    \begin{equation}
        \bm _eH_{L_O} = CDF(H_{L_O})
    \end{equation}
    The returned object, let's say $_eH_{L_O}$ is passed with $H_{L_I}$ as parameter to compute the matching histogram function. It's can be obtained using the inverse CDF.
    \begin{equation}
        \bm _aH_{L_O} = \widehat{CDF}_{L_I}^{-1}(H_{L_I}, _eH_{L_O})
        \label{inv:cdf}
    \end{equation}
    
    From \ref{inv:cdf}, we have as result the histogram of $L_O$ component approximately equal to $H_{L_I}$. Using this component, we can reconstruct the original image $O$, replacing $H_{L_O}$ to $_aH_{L_O}$. Finaly, we have the enhanced original image formed by $O = (_aH_{L_O}, a_I, b_I)$ and converted back to RGB space color $O = (r_O, a_O, b_O)$.
    
    \subsection{Evaluation Metrics}
    
    
    
    
    
    
    
    
    
    
    
    
    
    
    
    
    \section{RESULTS}
    \label{sec:pagestyle}
    
    
    \section{DISCUSSION}
    \label{sec:typestyle}
    
    
    \bibliographystyle{agsm}
    \bibliography{../references}
    
    \end{document}
